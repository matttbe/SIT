\typeout{------------------------------------------------------------------}
\typeout{} 
\typeout{        Fichier de base modifie par : Matth: 20 nov 2012} 
\typeout{                   sous licence GNU-GPL} 
\typeout{}
\typeout{------------------------------------------------------------------}

% Classe générale du document
   %documentclass[12pt,a4paper]{article} % .10pt, 11pt, 12pt : taille de la police principale (10 par défaut)
                 % .a4paper, letterpaper,... : délimite la taille du papier. (letterpaper par défaut)
	         % .fleqn : aligne les formules mathématiques à gauche au lieu de les centrer.
		 % .leqno : place la numérotation des formules à gauche plutôt qu'à droite.
		 % .twocolumn : demande à LATEX de formater le texte sur deux colonnes.
		 % .twoside, oneside indique si la sortie se fera en recto-verso ou en recto simple.
		 % .landscape, mais il faut mettre en commentaire (ou modifier) toutes les dimmensions

% Importation de packages divers
%   \NeedsTeXFormat{LaTeX2e} 
   \usepackage[T1]{fontenc}
   \usepackage[utf8x]{inputenc}		% utilisation des caractères 8 bits en Unix (codage ISO 8859-1)
  %\usepackage[latin1]{inputenc}	% utilisation des caractères pour Linux2
   \usepackage[usenames]{color}
   \usepackage{fancyhdr}
   \usepackage{lastpage}                % pour l'affichage du n° de la dernière page.
   \usepackage{lmodern}
   \usepackage{multirow}                % pour l'utilisation de figures ``noyées'' dans le texte
   \usepackage{xspace}			% package pour babel
   \usepackage[english]{babel}         % Utilisation du français (nom des sections, césure, ponctuation,...)
   \usepackage[english]{varioref}        % vpageref
   \usepackage{amsmath,amsthm,amssymb}  % Utilisation de certains packages de AMS (cf. belles équations)
   \usepackage{endnotes}                % Pour l'utilisation des notes en fin de documents
   \usepackage{verbatim}                % Pour l'insertion de fichier en mode verbatim
   \usepackage{portland}		% pour l'utilisation de \portrait et de \landscape sur une page
   \usepackage[pdftex]{graphicx}        % [pdftex] si utilisation d'images jgp,...
                                        % [dvips]  si utilisation d'images bmp,...
   \usepackage{pdfpages}		% Inclure des pages de pdf
   \usepackage{pdflscape}		% rotate => begin{landscape} ... \end{landscape}
   \usepackage{setspace}		% Pour définir un interligne
   \usepackage{tkz-orm}
   \usepackage[bottom]{footmisc}        % Footnote at the bottom
%   \usepackage[cyr]{aeguill}		% Pour les guillemets à la Française
   \usepackage{eurosym}			% Pour les Euro
   \usepackage{url}
    \urlstyle{sf}
   \usepackage[backgroundcolor=yellow]{todonotes} %% todonotes: \listoftodos & \todo{Some note or other.} & \missingfigure{}
	
   \renewcommand{\contentsname}{Sommaire} % si tableofcontents au début
   \newcommand{\Numero}{\No}
   \newcommand{\numero}{\no}
%   \newcommand{\fup}[1]{\up{#1}}

   \DeclareGraphicsExtensions{.jpg,.pdf,.mps,.png}       % déclaration d'extensions  pour les images
   %\input xy                            % pour le package xy (construction de diagramme)
   %\xyoption{all}

% Dimensions de la page :       	

  %%%%%%%%%%%%%%%%%%%%%%%%%%%%%%%%%%%%%%%%  0
  %   |                                  %
  %---+----------------------------------%  1
  %   | +----------------------------+   %  2
  %   | |          en-tête           |   %
  %   | +----------------------------+   %  3
  %   | +----------------------------+   %  4
  %   | |                            |   %       Remarques : 
  %   | |                            |   %        . distance de '0' à '1' : un pouce + \voffset
  %   | |                            |   %        . distance de 'a' à 'b' : un pouce + \hoffset
  %   | |           texte            |   %
  %   | |                            |   %
  %   | |                            |   %
  %   | |                            |   %
  %   | +----------------------------+   %  5
  %   | +----------------------------+   %
  %   | |         bas de page        |   %
  %   | +----------------------------+   %  6
  %%%%%%%%%%%%%%%%%%%%%%%%%%%%%%%%%%%%%%%%
  %a  b c                            d  e

%    % général
%      \voffset       0mm    % pour descendre (si positif) ou remonter (si négatif) le tout
%      \hoffset       0mm    % pour agrandir (si positif) ou diminuer (si négatif) la marge gauche (distance 'a' 'b')
%      \oddsidemargin 0mm   % 5pt  % distance de 'b' à 'c'
%     \evensidemargin 25mm  % 15pt % distance de 'd' à 'e'
%    % texte
%      \headsep       25pt   % distance de '3' à '4', la distance entre l'en-tête et le texte
%     \textheight    220mm  % distance de '4' à '5', pour déterminer la hauteur du texte
%     \textwidth     160mm  % distance de 'c' à 'd' 
%    % en-tête
%     \topmargin     0pt    % distance de '1' à '2', pour descendre (si positif) ou remonter (si négatif) le tout
%     \headheight    15pt   % distance de '2' à '3', doit être > 14.49999
%    % bas de page
%     \footskip      15mm   % 30pt % distance de '5' à '6', la distance entre le texte et le bas de page
     % space for the footnode
%    \setlength{\skip\footins}{1cm}

\usepackage[top=2.5cm, bottom=2.5cm, left=2.5cm, right=2.5cm]{geometry}

% (Re)définitions diverses

  % redéfinition de l'affichage des titres de section dans l'en-tête ou le bas de page
    % remarques :
    %  .affichage du numéro (2)    : \thesection 
    %  .affichage du nom (Section) : \sectionname
   % \renewcommand{\sectionmark}[1]{\markright{\thesection.\ #1}}   % 2.2. nom de la section 2.2
   % \renewcommand{\thesection}{\arabic{section}}		% II nom de la section 0.2


  % des couleurs...                   (utilisation avec par ex. \textcolor{webdarkblue}{...})
   \definecolor{codeBlue}{rgb}{0,0,1}
   \definecolor{webred}{rgb}{0.5,0,0}
   \definecolor{codeGreen}{rgb}{0,0.5,0}
   \definecolor{codeGrey}{rgb}{0.6,0.6,0.6}
   \definecolor{webdarkblue}{rgb}{0,0,0.4}
   \definecolor{webgreen}{rgb}{0,0.3,0}
   \definecolor{webblue}{rgb}{0,0,0.8}
   \definecolor{orange}{rgb}{0.7,0.1,0.1}

  % utilisation de caption, label,... pour autre chose qu'une figure
        %%%% debut macro %%%%
   \makeatletter
   \def\captionof#1#2{{\def\@captype{#1}#2}}
   \makeatother
        %%%% fin macro %%%%


% remarques : 
%  . pour mettre la date                  : \today
%  . pour mettre le nom de la section     : \rightmark
%  . pour mettre le numéro de page        : \thepage
%  . pour mettre le nombre de pages total : \pageref{LastPage}  (mais l'écrit en rouge vu que c'est une réf.)
%  . insertion d'une image                : \setlength{\unitlength}{1mm}
%                                             \begin{picture}(0,0)
%                                                \put(5,0){\includegraphics[scale=x.x]{xxx.xxx}}
%                                             \end{picture}

% Pour les guillemets à  la Française
\newcommand{\fermerguillemets}{\unskip\kern.15em\symbol{20}}
\newcommand{\ouvrerguillemets}{\symbol{19}\ignorespaces\kern.15em}
% \let »=\fermerguillemets
% \let« =\ouvrerguillemets

% Pour changer l'icone des puces : à placer juste avant une liste
 %   \renewcommand\labelitemi{\textbullet}	% Style boulet :)