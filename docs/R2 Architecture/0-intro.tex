At the begining of this phase, we just received the feedback of the previous one, so we knew the pros and cons of our work.
Looking at these informations, we decided to improve ourselves by taking the remarks in account.
The goals of this part of the job was to go further in the modelisation of the architecture of the program.
To do this, we had to create some diagrams : 
\begin{itemize}
\item A class diagram, showing the differents objects, their attributes, their main methods and the relations linking them.
\item Some sequences diagrams, to illustrate the most important user activities by displaying the interactions between the objects in several defined scenarios.
\item An ORM diagram, to show graphically the fields of the database and its architecture.
\item A box pointer diagram, to explain the way that the modules of the website are working together sequentially on a timeline.
\end{itemize}
Doing all theses diagrams will allow us to have a better overview of the project. So we will have a more precise idea on how the website will run. 
Two others requiered elements in this report were to detail the physical architecture and the framework that we chose and to argue the choice.
We had to do research to justify our choice and have a look on the physical architecture and how the chosen framework imposes it.