\subsection*{Advantages of the framework}

We chose \emph{Ruby on Rails (RoR)} as framework for the project. After some research about existing frameworks, we found that RoR gives us many advantages. We have listed these pros here :

\begin{itemize}
   \item RoR constraints us to use the MVC design pattern which we are comfortable with.
   \item RoR forces us to apply good practices like \emph{Don't repeat yourself}.
   \item RoR has conventions over configurations that allow us to waste less time configuring our project and making choices.
   \item RoR has an integrated structure. It allows us to have a clean code through the all project.
   \item RoR make easier to manipulate DB. So, it allows to reuse the same DB throughout all devices and platforms (mobile apps, etc.).
   \item RoR already contains development environment like \emph{production}, \emph{test} and \emph{development}. It makes the deployment easier.
   \item RoR provides protection tools against the most used security attacks. It's important in this case because the platform will contains some confidential informations about the users.
   \item RoR has a large active community of users. Then it is easier to find help and documentation. Moreover there
   are some \emph{gems} availale to allow us to add some features very quickly (like some extensions proposed). 
   \item There's a lot of offers to host and deploy a RoR project. Moreover some are free to use and 
   propose all tools for a middle-sized website.  In addition, we found some maintenance tools like failures log or security tools.
\end{itemize}

In addition, lot of companies are looking for RoR developers. So, we're very interested in learning it. Moreover, 
some members of the group knew it already so they will be able to guide the rest of the group.

\subsection*{Comparisons of the frameworks}

We chose Ruby on rails rather than other frameworks for multiples reasons.  
If we look at the languages used by the frameworks, we made the choice not to use PHP 
and Java.  PHP become less and less used by companies for web applications because it not support new 
web features and a clear model programming.  For Java language, we didn't find a very good framework 
that allows to use all possiilites of Java.  Some years ago, Google develloped GWT, a good framework 
that transformed Java in Javascript but the programming model was too heavy to offer a new opportunity to 
use Java for web applications.  Moreover, with a Java web applications, we must use a server that allows 
Java and we did't find a good free offer for this kind of service.

In the languages that are not known by at least one member of the group, we found Smalltalk with 
Seaside framework.  After a first look at the Seaside website, we found a small community with not 
enough documentation and plugins.  Moreover, we didn't find a lot of websites based on Seaside.

We also looked the advantages of Django and really hesitated between Django and RoR. But we learned that
the Django documentation is smaller than the RoR one. So it's more difficult to discover it according to some articles found on the Internet. Moreover, we have to learn python this semester for other 
courses so it seemed more interesting to learn a new language.
